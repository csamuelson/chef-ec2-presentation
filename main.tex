% $Header: /cvsroot/latex-beamer/latex-beamer/solutions/conference-talks/conference-ornate-20min.en.tex,v 1.7 2007/01/28 20:48:23 tantau Exp $

\documentclass{beamer}

% This file is a solution template for:

% - Talk at a conference/colloquium.
% - Talk length is about 20min.
% - Style is ornate.



% Copyright 2011 Christopher J. Samuelson

\mode<presentation>
{
  \usetheme{Warsaw}
  % or ...

  \setbeamercovered{transparent}
  % or whatever (possibly just delete it)
  \useinnertheme{rectangles}
}


\usepackage[english]{babel}
% or whatever

\usepackage[latin1]{inputenc}
% or whatever

\usepackage{times}
\usepackage[T1]{fontenc}
% Or whatever. Note that the encoding and the font should match. If T1
% does not look nice, try deleting the line with the fontenc.


\title % (optional, use only with long paper titles)
{How to get Rails apps into the cloud with Chef}

\subtitle
{Rails + Chef + Amazon Web Services}

\author[@chrissamuelson] % (optional, use only with lots of authors)
{Chris Samuelson}
% - Give the names in the same order as the appear in the paper.
% - Use the \inst{?} command only if the authors have different
%   affiliation.


\date % (optional, should be abbreviation of conference name)
{\today}
% - Either use conference name or its abbreviation.
% - Not really informative to the audience, more for people (including
%   yourself) who are reading the slides online

\subject{Cloud Computing}
% This is only inserted into the PDF information catalog. Can be left
% out. 



% If you have a file called "university-logo-filename.xxx", where xxx
% is a graphic format that can be processed by latex or pdflatex,
% resp., then you can add a logo as follows:

% \pgfdeclareimage[height=0.5cm]{university-logo}{university-logo-filename}
% \logo{\pgfuseimage{university-logo}}



% Delete this, if you do not want the table of contents to pop up at
% the beginning of each subsection:
\AtBeginSubsection[]
{
  \begin{frame}<beamer>{Outline}
    \tableofcontents[currentsection,currentsubsection]
  \end{frame}
}


% If you wish to uncover everything in a step-wise fashion, uncomment
% the following command: 

%\beamerdefaultoverlayspecification{<+->}


\begin{document}

\begin{frame}
  \titlepage
\end{frame}

\begin{frame}{Outline}
  \tableofcontents
  % You might wish to add the option [pausesections]
\end{frame}


% Structuring a talk is a difficult task and the following structure
% may not be suitable. Here are some rules that apply for this
% solution: 

% - Exactly two or three sections (other than the summary).
% - At *most* three subsections per section.
% - Talk about 30s to 2min per frame. So there should be between about
%   15 and 30 frames, all told.

% - A conference audience is likely to know very little of what you
%   are going to talk about. So *simplify*!
% - In a 20min talk, getting the main ideas across is hard
%   enough. Leave out details, even if it means being less precise than
%   you think necessary.
% - If you omit details that are vital to the proof/implementation,
%   just say so once. Everybody will be happy with that.

\section{Introduction}

\subsection{Introduction}

\begin{frame}{Who - Chris Samuelson}
  % - A title should summarize the slide in an understandable fashion
  %   for anyone how does not follow everything on the slide itself.
  
  \begin{itemize}
  \item
    SwingPal
    \pause
  \item
    We use the tools I'm going to talk about.
  \end{itemize}
\end{frame}

\begin{frame}{What}
Describe running an app in the cloud by working through a concrete example using:
\pause
\begin{itemize}
  \item
    Ruby on Rails
    \pause
  \item
    Capistrano
    \pause
  \item
    Chef
    \pause
  \item
    Amazon Web Services
\end{itemize}
\end{frame}


\subsection{Commodity Infrastructure}

\begin{frame}{Commodity Infrastructure}
  \begin{block}{What do I mean by Commodity Infrastructure?}
    \begin{itemize}
    \item Pay for what you use computing
    \item Similar products let you migrate between platforms
    \item Many common tasks can be done using pay per use services
    \end{itemize}
  \end{block}
\end{frame}


\begin{frame}{Isn't that just a VPS with a new name?}
\pause
\begin{block}{Amazon, Rackspace Cloud, etc - compute units resemble VPS}
  \begin{itemize}
  \item No setup fees or long-term contracts
  \item Granular charges for resources (hourly vs monthly)
  \item Able to increase or decrease resources quickly
  \end{itemize}
\end{block}
\pause
\begin{block}{Google, Salesforce, etc - managed hosting meets SaaS}
  \begin{itemize}
  \item No server software to manage
  \item Automatically manages capacity
  \item Pay only for what you use
  \item Many common tasks are services - image processing
  \end{itemize}
\end{block}
\end{frame}

\begin{frame}{Isn't that just a VPS with a new name?}
  \begin{block}{Bottom Line:}
    You could use a service like Amazon's EC2 just like a VPS, but then you'd be missing the whole point.
  \end{block}
\end{frame}

\subsection{Not just about the hardware...}
\begin{frame}{Configuration Management and Deployment Automation}
  \begin{block}{Question:}
    Now that you have freely scalable resources what are you going to do with them?
  \end{block}
  \begin{block}{Answer:}
    Configuration Management, and Deployment Automation to the rescue!
  \end{block}
\end{frame}

\begin{frame}{Configuration Management}
  \begin{itemize}
  \item No need to maintain 'Golden Image'
  \item Keep software packages up to date more easily
  \item Know that you will be able to turn up a new server at any time
  \end{itemize}
\end{frame}

\begin{frame}{Deployment Automation}
  \begin{itemize}
  \item Manage software releases
  \item Know that your software deployment is documented
  \item Know that you will be able to turn up a new server at any time
  \end{itemize}
\end{frame}

\section{Example Stack}

\subsection{Components}

\begin{frame}{Components}
  \begin{columns}[t]
    \begin{column}{4cm}
      \begin{itemize}
      \uncover<1->{\item Ruby on Rails}
      \uncover<2->{\item Capistrano}
      \uncover<3->{\item Chef}
      \uncover<4->{\item Amazon Web Services}
      \end{itemize}
    \end{column}
    \begin{column}{6cm}
      \only<1-4>{\begin{itemize}}
      \only<1>{\item Web Application Framework}
      \only<1>{\item Model-View-Controller Architecture}
      \only<1>{\item RubyGems package manager}
      \only<2>{\item Deployment Automation}
      \only<2>{\item Deploys code and database changes}
      \only<2>{\item Can be used for other remote admin tasks}
      \only<3>{\item Configuration Management}
      \only<3>{\item Mutual authentication and encryption}
      \only<3>{\item Easily configure applications that require knowledge about your entire infrastructure}
      \only<4>{\item Infrastructure web services platform in the cloud}
      \only<4>{\begin{itemize}}
      \only<4>{\item EC2 - Elastic Compute Cloud}
      \only<4>{\item S3 - Simple Storage Service}
      \only<4>{\item RDS - Relational Database Service}
      \only<4>{\end{itemize}}
      \only<1-4>{\end{itemize}}
    \end{column}
  \end{columns}
\end{frame}

\begin{frame}{Make Titles Informative.}
\end{frame}

\begin{frame}{Make Titles Informative.}
\end{frame}


\subsection{Basic Ideas for Proofs/Implementation}

\begin{frame}{Make Titles Informative.}
\end{frame}

\begin{frame}{Make Titles Informative.}
\end{frame}

\begin{frame}{Make Titles Informative.}
\end{frame}



\section*{Summary}

\begin{frame}{Summary}

  % Keep the summary *very short*.
  \begin{itemize}
  \item
    The \alert{first main message} of your talk in one or two lines.
  \item
    The \alert{second main message} of your talk in one or two lines.
  \item
    Perhaps a \alert{third message}, but not more than that.
  \end{itemize}
  
  % The following outlook is optional.
  \vskip0pt plus.5fill
  \begin{itemize}
  \item
    Outlook
    \begin{itemize}
    \item
      Something you haven't solved.
    \item
      Something else you haven't solved.
    \end{itemize}
  \end{itemize}
\end{frame}



% All of the following is optional and typically not needed. 
\appendix
\section<presentation>*{\appendixname}
\subsection<presentation>*{For Further Reading}

\begin{frame}[allowframebreaks]
  \frametitle<presentation>{For Further Reading}
    
  \begin{thebibliography}{10}
    
  \beamertemplatebookbibitems
  % Start with overview books.

  \bibitem{Author1990}
    A.~Author.
    \newblock {\em Handbook of Everything}.
    \newblock Some Press, 1990.
 
    
  \beamertemplatearticlebibitems
  % Followed by interesting articles. Keep the list short. 

  \bibitem{Someone2000}
    S.~Someone.
    \newblock On this and that.
    \newblock {\em Journal of This and That}, 2(1):50--100,
    2000.
  \end{thebibliography}
\end{frame}

\end{document}


